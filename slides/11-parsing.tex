%% -*- coding: utf-8 -*-
\documentclass{beamer}

%% -*- coding: utf-8 -*-
\usetheme{Boadilla} % default
\useoutertheme{infolines}
\setbeamertemplate{navigation symbols}{} 

\usepackage{etex}
\usepackage{alltt}
\usepackage{pifont}
\usepackage{color}
\usepackage[utf8]{inputenc}
%\usepackage{german}
\usepackage{listings}
\lstset{language=Haskell}
\lstset{sensitive=true}
\usepackage{hyperref}
\hypersetup{colorlinks=true}
\usepackage[final]{pdfpages}
\usepackage{url}
\usepackage{arydshln} % dashed lines
\usepackage{tikz}
\usepackage{mathpartir}

\DeclareUnicodeCharacter{3BB}{\ensuremath{\lambda}}


\newcommand\cmark{\ding{51}}
\newcommand\xmark{\ding{55}}

\newcommand{\nat}{\mathbf{N}}

\usepackage[all]{xy}

%% new arrow tip for xy
\newdir{|>}{!/4.5pt/@{|}*:(1,-.2)@^{>}*:(1,+.2)@_{>}}

\newcommand\cid[1]{\textup{\textbf{#1}}} % class names
\newcommand\kw[1]{\textup{\textbf{#1}}}  % key words
\newcommand\tid[1]{\textup{\textsf{#1}}} % type names
\newcommand\vid[1]{\textup{\texttt{#1}}} % value names
\newcommand\Mid[1]{\textup{\texttt{#1}}} % method names

\newcommand\TODO[1][]{{\color{red}{\textbf{TODO: #1}}}}

\newcommand\String[1]{\texttt{\dq{}#1\dq{}}}

\newcommand\ClassHead[1]{%
  \ensuremath{\begin{array}{|l|}
      \hline
      \cid{#1}
      \\\hline
    \end{array}}}
\newcommand\AbstractClass[2]{%
  \ensuremath{\begin{array}{|l|}
      \hline
      \cid{\textit{#1}}
      \\\hline
      #2
      \hline
    \end{array}}}
\newcommand\Class[2]{%
  \ensuremath{\begin{array}{|l|}
      \hline
      \cid{#1}
      \\\hline
      #2
      \hline
    \end{array}}}
\newcommand\Attribute[3][black]{\textcolor{#1}{\Param{#2}{#3}}\\}
\newcommand\Methods{\hline}
\newcommand\MethodSig[3]{\Mid{#2} (#3): \,\tid{#1}\\}
\newcommand\CtorSig[2]{\Mid{#1} (#2)\\}
\newcommand\AbstractMethodSig[3]{\Mid{\textit{#2}} (#3): \,\tid{#1}\\}
\newcommand\Param[2]{\vid{#2}:~\tid{#1}}

\lstset{%
  frame=single,
  xleftmargin=2pt,
  stepnumber=1,
  numbers=left,
  numbersep=5pt,
  numberstyle=\ttfamily\tiny\color[gray]{0.3},
  belowcaptionskip=\bigskipamount,
  captionpos=b,
  escapeinside={*'}{'*},
  % language=java,
  tabsize=2,
  emphstyle={\bf},
  commentstyle=\mdseries\it,
  stringstyle=\mdseries\rmfamily,
  showspaces=false,
  showtabs=false,
  keywordstyle=\bfseries,
  columns=fullflexible,
  basicstyle=\footnotesize\CodeFont,
  showstringspaces=false,
  morecomment=[l]\%,
  rangeprefix=////,
  includerangemarker=false,
}

\newcommand\CodeFont{\sffamily}

\definecolor{lightred}{rgb}{0.8,0,0}
\definecolor{darkgreen}{rgb}{0,0.5,0}
\definecolor{darkblue}{rgb}{0,0,0.5}

\newcommand\highlight[1]{\textcolor{blue}{\emph{#1}}}
\newcommand\GenClass[2]{\cid{#1}\texttt{<}\cid{#2}\texttt{>}}

\newcommand\Colored[3]{\alt<#1>{\textcolor{#2}{#3}}{#3}}

\newcommand\nt[1]{\ensuremath{\langle#1\rangle}}

\newcommand{\free}{\operatorname{free}}
\newcommand{\bound}{\operatorname{bound}}
\newcommand{\var}{\operatorname{var}}
\newcommand\VSPBLS{\vspace{-\baselineskip}}

\newcommand\IF{\textit{IF}}
\newcommand\TRUE{\textit{TRUE}}
\newcommand\FALSE{\textit{FALSE}}

\newcommand\IFZ{\textit{IF0}}
\newcommand\ZERO{\textit{ZERO}}
\newcommand\SUCC{\textit{SUCC}}
\newcommand\ADD{\textit{ADD}}
\newcommand\SUB{\textit{SUB}}
\newcommand\MULT{\textit{MULT}}
\newcommand\DIV{\textit{DIV}}

\newcommand\PAIR{\textit{PAIR}}
\newcommand\FST{\textit{FST}}
\newcommand\SND{\textit{SND}}

\newcommand\CASE{\textit{CASE}}
\newcommand\LEFT{\textit{LEFT}}
\newcommand\RIGHT{\textit{RIGHT}}

\newcommand\Encode[1]{\lceil#1\rceil}
\newcommand\Reduce{\stackrel\ast\rightarrow_\beta}

\newcommand\Nat{\textit{Nat}}
\newcommand\Bool{\textit{Bool}}
\newcommand\Pair{\textit{Pair}}
\newcommand\Tfun[1]{#1\to}

\newcommand\Tenv{A}
\newcommand\Lam[1]{\lambda#1.}
\newcommand\App[1]{#1\,}
\newcommand\Succ{\textit{SUCC}\,}
\newcommand\Let[2]{\textit{let}\,#1=#2\,\textit{in}\,}

\newcommand\calE{\mathcal{E}}
\newcommand\calU{\mathcal{U}}
\newcommand\calP{\mathcal{P}}
\newcommand\calW{\mathcal{W}}

\newcommand\GEN{\textit{gen}}
\newcommand\EFV[1]{\textit{fv} (#1)}
\newcommand\Dom[1]{\textit{dom} (#1)}

%%% Local Variables: 
%%% mode: latex
%%% TeX-master: nil
%%% End: 

%%% frontmatter
%% -*- coding: utf-8 -*-

\title{Functional Programming}
\subtitle{Introduction}

\author[Peter Thiemann]{Prof. Dr. Peter Thiemann}
\institute[Univ. Freiburg]{Albert-Ludwigs-Universität Freiburg, Germany}
\date{SS 2021}


\subtitle
{Parsing}

\begin{document}

\begin{frame}
  \titlepage
\end{frame}

%\begin{frame}
%  \frametitle{Outline}
%  \tableofcontents
  % You might wish to add the option [pausesections]
%\end{frame}


\begin{frame}[fragile]
  \frametitle{Recall the expression language}
\begin{block}<+->{Definition}
\begin{lstlisting}
data Term  = Con Integer
           | Bin Term Op Term  
             deriving (Eq, Show)
           
data Op    = Add | Sub | Mul | Div
             deriving (Eq, Show)
\end{lstlisting}
\end{block}
\begin{alertblock}<+->{Parsing expressions}
  \begin{itemize}
  \item Read a string like \texttt{"3+42/6"}
  \item Recognize it as a valid term
  \item Return \texttt{Bin (Con 3) Add (Bin (Con 42) Div (Con 6))} 
  \end{itemize}
\end{alertblock}
\end{frame}             

\begin{frame}[fragile]
  \frametitle{Parsing}
\begin{block}{The type of a simple parser}
\begin{lstlisting}
type Parser token result = [token] -> [(result, [token])]
\end{lstlisting}  
\end{block}
\end{frame}             

\begin{frame}[fragile]
  \frametitle{Combinator parsing}
  \begin{block}{Primitive parsers}
\begin{lstlisting}
pempty :: Parser t r
succeed :: r -> Parser t r
satisfy :: (t -> Bool) -> Parser t t
msatisfy :: (t -> Maybe a) -> Parser t a
lit :: Eq t => t -> Parser t t
\end{lstlisting}
  \end{block}
\end{frame}

\begin{frame}[fragile]
  \frametitle{Combinator parsing II}
  \begin{block}{Combination of parsers}
\begin{lstlisting}
palt :: Parser t r -> Parser t r -> Parser t r
pseq :: Parser t (s -> r) -> Parser t s -> Parser t r
pmap :: (s -> r) -> Parser t s -> Parser t r
\end{lstlisting}
\end{block}     
\end{frame}             


\begin{frame}[fragile]
  \frametitle{A taste of compiler construction}
\begin{block}{A lexer}
 A lexer partitions the incoming list of
 characters into a list of tokens. A token is either a single symbol, 
 an identifier, or a number. Whitespace characters are removed.
\end{block}
\end{frame}     


\begin{frame}[fragile]
  \frametitle{Underlying concepts}
  \begin{alertblock}{Parsers have a rich structure}
    \begin{itemize}
    \item many concepts from category theory can be mapped to programming concepts
    \item parsing illustrates many of these concepts
    \end{itemize}
  \end{alertblock}
\end{frame}             


\begin{frame}[fragile]
  \frametitle{Functors}
  \begin{alertblock}{The functor class}
\begin{lstlisting}
class Functor f where
  fmap :: (a -> b) -> (f a -> f b)
\end{lstlisting}  
\end{alertblock}

\begin{exampleblock}{Instances}
  List, Maybe, IO, \dots
\end{exampleblock}
\begin{alertblock}{Functorial laws}
\begin{lstlisting}
fmap id_a == id_f_a
fmap (f . g) == fmap f . fmap g
\end{lstlisting}
\end{alertblock}
\end{frame}             

\begin{frame}[fragile]
  \frametitle{Parsing is \dots}
  \begin{block}{A functor}
    Check the functorial laws!
  \end{block}
  \begin{block}{A monad}
    Check the monad laws!
  \end{block}
  \begin{alertblock}{Consequence}
    Can use \texttt{do} notation for parsing!
  \end{alertblock}
\end{frame}


\begin{frame}[fragile]
  \frametitle{Applicative}
  \begin{alertblock}{Example 1: sequencing computation}
\begin{lstlisting}
sequence :: [IO a] -> IO [a]
sequence []       = return []
sequence (io:ios) = do x <- io
                       xs <- sequence ios
                       return (x:xs)
\end{lstlisting}  
  \end{alertblock}
  \begin{alertblock}{Alternative way}
\begin{lstlisting}
sequence []       = return []
sequence (io:ios) = return (:) `ap` io `ap` sequence ios

return :: Monad m => a -> m a
ap     :: Monad m => m (a -> b) -> m a -> m b
\end{lstlisting}
  \end{alertblock}
\end{frame}



\begin{frame}[fragile]
 \frametitle{Applicative}
 \begin{exampleblock}{Example 2: transposition}
\begin{lstlisting}
transpose :: [[a]] -> [[a]]
transpose [] = repeat []
transpose (xs:xss) = zipWith (:) xs (transpose xss)
\end{lstlisting}
 \end{exampleblock}
 \begin{alertblock}{Rewrite}
\begin{lstlisting}
transpose []       = repeat []
transpose (xs:xss) = repeat (:) `zapp` xs `zapp` transpose xss

zapp :: [a -> b] -> [a] -> [b]
zapp fs xs = zipWith ($) fs xs
\end{lstlisting}
 \end{alertblock}
\end{frame}             

\begin{frame}[fragile]
  \frametitle{Applicative Interpreter}
  \begin{block}{Standard interpretation}
\begin{lstlisting}
data Exp v
  = Var v
  | Val Int
  | Add (Exp v) (Exp v)

eval :: Exp v -> Env v -> Int
eval (Var v) env = fetch v env
eval (Val i) env = i
eval (Add e1 e2) env = eval e1 env + eval e2 env

type Env v = v -> Int
fetch :: v -> Env v -> Int
fetch v env = env v
\end{lstlisting} 
\end{block}
\end{frame}


\begin{frame}[fragile]
  \frametitle{Applicative Interpreter}
  \begin{alertblock}{Alternative implementation}
\begin{lstlisting}
eval' :: Exp v -> Env v -> Int
eval' (Var v) = fetch v
eval' (Val i) = const i
eval' (Add e1 e2) = const (+) `ess` (eval' e1) `ess` (eval' e2)

ess a b c = (a c) (b c)
\end{lstlisting} 
\end{alertblock}
\end{frame}

\begin{frame}[fragile]
  \frametitle{Applicative}
\begin{exampleblock}{Extract the common structure}
\begin{lstlisting}
class Functor f => Applicative f where
  pure  :: a -> f a
  (<*>) :: f (a -> b) -> f a -> f b
\end{lstlisting} 
\end{exampleblock}
\end{frame}            

\begin{frame}[fragile]
  \frametitle{Applicative}
  \begin{block}{Laws}
  \begin{itemize}       
  \item Identity
\begin{lstlisting}
pure id <*> v == v
\end{lstlisting}
  \item Composition
\begin{lstlisting}
pure (.) <*> u <*> v <*> w = u <*> (v <*> w)
\end{lstlisting}
  \item Homomorphism
\begin{lstlisting}
pure f <*> pure x = pure (f x)
\end{lstlisting}
  \item Interchange
\begin{lstlisting}
u <*> pure y = pure ($ y) <*> u
\end{lstlisting}
  \end{itemize} 
  \end{block}   
\end{frame}

\begin{frame}[fragile]
  \frametitle{Parsers are Applicative!}
  \begin{alertblock}{}
\begin{lstlisting}
instance Applicative (Parser' token) where
  pure = return
  (<*>) = ap

instance Alternative (Parser' token) where
  empty = mzero
  (<|>) = mplus
\end{lstlisting}
  \end{alertblock}
\end{frame}




\begin{frame}
  \frametitle{Wrapup}
  \begin{itemize}[<+->]
  \item what if there are multiple applicatives?
  \item they just compose (unlike monads)
  \item applicative do notation
  \item applicatives cannot express dependency
  \item enable more clever parsers
  \end{itemize}
  
\end{frame}


\end{document}


