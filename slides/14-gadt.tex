%% -*- coding: utf-8 -*-
\documentclass[pdftex,aspectratio=169]{beamer}

\usepackage[T1]{fontenc}
\usepackage[utf8]{inputenc}
\usepackage{beramono}

%% -*- coding: utf-8 -*-
\usetheme{Boadilla} % default
\useoutertheme{infolines}
\setbeamertemplate{navigation symbols}{} 

\usepackage{etex}
\usepackage{alltt}
\usepackage{pifont}
\usepackage{color}
\usepackage[utf8]{inputenc}
%\usepackage{german}
\usepackage{listings}
\lstset{language=Haskell}
\lstset{sensitive=true}
\usepackage{hyperref}
\hypersetup{colorlinks=true}
\usepackage[final]{pdfpages}
\usepackage{url}
\usepackage{arydshln} % dashed lines
\usepackage{tikz}
\usepackage{mathpartir}

\DeclareUnicodeCharacter{3BB}{\ensuremath{\lambda}}


\newcommand\cmark{\ding{51}}
\newcommand\xmark{\ding{55}}

\newcommand{\nat}{\mathbf{N}}

\usepackage[all]{xy}

%% new arrow tip for xy
\newdir{|>}{!/4.5pt/@{|}*:(1,-.2)@^{>}*:(1,+.2)@_{>}}

\newcommand\cid[1]{\textup{\textbf{#1}}} % class names
\newcommand\kw[1]{\textup{\textbf{#1}}}  % key words
\newcommand\tid[1]{\textup{\textsf{#1}}} % type names
\newcommand\vid[1]{\textup{\texttt{#1}}} % value names
\newcommand\Mid[1]{\textup{\texttt{#1}}} % method names

\newcommand\TODO[1][]{{\color{red}{\textbf{TODO: #1}}}}

\newcommand\String[1]{\texttt{\dq{}#1\dq{}}}

\newcommand\ClassHead[1]{%
  \ensuremath{\begin{array}{|l|}
      \hline
      \cid{#1}
      \\\hline
    \end{array}}}
\newcommand\AbstractClass[2]{%
  \ensuremath{\begin{array}{|l|}
      \hline
      \cid{\textit{#1}}
      \\\hline
      #2
      \hline
    \end{array}}}
\newcommand\Class[2]{%
  \ensuremath{\begin{array}{|l|}
      \hline
      \cid{#1}
      \\\hline
      #2
      \hline
    \end{array}}}
\newcommand\Attribute[3][black]{\textcolor{#1}{\Param{#2}{#3}}\\}
\newcommand\Methods{\hline}
\newcommand\MethodSig[3]{\Mid{#2} (#3): \,\tid{#1}\\}
\newcommand\CtorSig[2]{\Mid{#1} (#2)\\}
\newcommand\AbstractMethodSig[3]{\Mid{\textit{#2}} (#3): \,\tid{#1}\\}
\newcommand\Param[2]{\vid{#2}:~\tid{#1}}

\lstset{%
  frame=single,
  xleftmargin=2pt,
  stepnumber=1,
  numbers=left,
  numbersep=5pt,
  numberstyle=\ttfamily\tiny\color[gray]{0.3},
  belowcaptionskip=\bigskipamount,
  captionpos=b,
  escapeinside={*'}{'*},
  % language=java,
  tabsize=2,
  emphstyle={\bf},
  commentstyle=\mdseries\it,
  stringstyle=\mdseries\rmfamily,
  showspaces=false,
  showtabs=false,
  keywordstyle=\bfseries,
  columns=fullflexible,
  basicstyle=\footnotesize\CodeFont,
  showstringspaces=false,
  morecomment=[l]\%,
  rangeprefix=////,
  includerangemarker=false,
}

\newcommand\CodeFont{\sffamily}

\definecolor{lightred}{rgb}{0.8,0,0}
\definecolor{darkgreen}{rgb}{0,0.5,0}
\definecolor{darkblue}{rgb}{0,0,0.5}

\newcommand\highlight[1]{\textcolor{blue}{\emph{#1}}}
\newcommand\GenClass[2]{\cid{#1}\texttt{<}\cid{#2}\texttt{>}}

\newcommand\Colored[3]{\alt<#1>{\textcolor{#2}{#3}}{#3}}

\newcommand\nt[1]{\ensuremath{\langle#1\rangle}}

\newcommand{\free}{\operatorname{free}}
\newcommand{\bound}{\operatorname{bound}}
\newcommand{\var}{\operatorname{var}}
\newcommand\VSPBLS{\vspace{-\baselineskip}}

\newcommand\IF{\textit{IF}}
\newcommand\TRUE{\textit{TRUE}}
\newcommand\FALSE{\textit{FALSE}}

\newcommand\IFZ{\textit{IF0}}
\newcommand\ZERO{\textit{ZERO}}
\newcommand\SUCC{\textit{SUCC}}
\newcommand\ADD{\textit{ADD}}
\newcommand\SUB{\textit{SUB}}
\newcommand\MULT{\textit{MULT}}
\newcommand\DIV{\textit{DIV}}

\newcommand\PAIR{\textit{PAIR}}
\newcommand\FST{\textit{FST}}
\newcommand\SND{\textit{SND}}

\newcommand\CASE{\textit{CASE}}
\newcommand\LEFT{\textit{LEFT}}
\newcommand\RIGHT{\textit{RIGHT}}

\newcommand\Encode[1]{\lceil#1\rceil}
\newcommand\Reduce{\stackrel\ast\rightarrow_\beta}

\newcommand\Nat{\textit{Nat}}
\newcommand\Bool{\textit{Bool}}
\newcommand\Pair{\textit{Pair}}
\newcommand\Tfun[1]{#1\to}

\newcommand\Tenv{A}
\newcommand\Lam[1]{\lambda#1.}
\newcommand\App[1]{#1\,}
\newcommand\Succ{\textit{SUCC}\,}
\newcommand\Let[2]{\textit{let}\,#1=#2\,\textit{in}\,}

\newcommand\calE{\mathcal{E}}
\newcommand\calU{\mathcal{U}}
\newcommand\calP{\mathcal{P}}
\newcommand\calW{\mathcal{W}}

\newcommand\GEN{\textit{gen}}
\newcommand\EFV[1]{\textit{fv} (#1)}
\newcommand\Dom[1]{\textit{dom} (#1)}

%%% Local Variables: 
%%% mode: latex
%%% TeX-master: nil
%%% End: 

%%% frontmatter
%% -*- coding: utf-8 -*-

\title{Functional Programming}
\subtitle{Introduction}

\author[Peter Thiemann]{Prof. Dr. Peter Thiemann}
\institute[Univ. Freiburg]{Albert-Ludwigs-Universität Freiburg, Germany}
\date{SS 2021}


\author[Gabriel Radanne]{Dr. Gabriel Radanne}
\subtitle
{GADT: Generalized Algebraic DataType}

\setbeamercovered{transparent=15}

\renewcommand\CodeFont{\ttfamily}
\lstset{%
  frame=none,
  language=Haskell,
  escapeinside={--$}{$}
}

\begin{document}

\begin{frame}
  \titlepage
\end{frame}

\begin{frame}[fragile]
  \frametitle{Interpreters, again}
  \begin{block}{Language definition}
    \begin{lstlisting}
data Term  = I Integer
           | Add Term Term
             deriving (Eq, Show)
  \end{lstlisting}
\end{block}
\end{frame}

\begin{frame}[fragile]
  \frametitle{Interpreters, again}
  \begin{block}{Evaluation}
    \begin{lstlisting}
eval             :: Term -> Integer
eval (I n)        = n
eval (Add t op u) = eval t + eval u
  \end{lstlisting}
\end{block}
\end{frame}


\begin{frame}[fragile]
  \frametitle{A language with multiple types}
  \begin{block}{Language definition}
    \begin{lstlisting}
data Term  = I Integer
           | B Bool
           | Add Term Term
           | Eq Term Term
             deriving (Eq, Show)
  \end{lstlisting}
\end{block}
\end{frame}

\begin{frame}[fragile]
  \frametitle{A language with multiple types}
  \begin{block}{Evaluation}
    \begin{lstlisting}
type Value = Int Integer | Bool Bool
             deriving Show
      
eval           :: Term -> Value
eval (I n)      = Int n
eval (B b)      = Bool b
eval (Add t t') = case (eval t, eval t') of
                    (Int i, Int i2) -> Int (i + i2)
eval (Eq t t')  = case (eval t, eval t') of
                    (Int i, Int i2) -> Bool (i == i2)
                    (Bool i, Bool i2) -> Bool (i == i2)
  \end{lstlisting}
\end{block}
\end{frame}

\begin{frame}{Issues with our interpreter}
  \begin{itemize}[<+->]
\item The interpreter can fail because of type mismatch.
\item We need to deal with failures manually.
\item The more values we have in the language, the more complicated it becomes.
\item The Haskell type system does not help us.
\end{itemize}
\end{frame}

\begin{frame}[fragile]
  \frametitle{GADTs to the rescue!}

  \begin{block}{Algebraic Data Type}
    \begin{lstlisting}
data Maybe a =
    Nothing
  | Just a
\end{lstlisting}

  \end{block}
\end{frame}

\begin{frame}[fragile]
  \frametitle{GADTs to the rescue!}

  \begin{block}{Generalized Algebraic Data Type}
    \begin{lstlisting}
{-# LANGUAGE GADTs #-}

data Maybe a where
  Nothing :: Maybe a
  Just :: a -> Maybe a
\end{lstlisting}

  \end{block}\pause
  \begin{itemize}
  \item Now we also specify the return type of the data constructors!
  \item We cannot change the type constructor \lstinline{Maybe}, but
    its arguments may vary
  \end{itemize}
\end{frame}

\begin{frame}[fragile]
  \frametitle{Language definition with GADTs}
  \begin{block}{}
    \begin{lstlisting}
data Term =
    I Integer
  | B Bool
  | Add Term Term
  | Eq Term Term      
    \end{lstlisting}
  \end{block}
\end{frame}

\begin{frame}[fragile]
  \frametitle{Language definition with GADTs}
  \begin{block}{}
    \begin{lstlisting}
data Term where
  I :: Integer -> Term
  B :: Bool -> Term
  Add :: Term -> Term -> Term
  Eq :: Term -> Term -> Term   
    \end{lstlisting}
  \end{block}
\end{frame}

\begin{frame}[fragile]
  \frametitle{Language definition with GADTs}
  \begin{block}{}
    \begin{lstlisting}
data Term a where
  I :: Integer -> Term Integer
  B :: Bool -> Term Bool
  Add :: Term (?) -> Term (?) -> Term (?)
  Eq :: Term (?) -> Term (?) -> Term (?)
    \end{lstlisting}
  \end{block}
\end{frame}

\begin{frame}[fragile]
  \frametitle{Language definition with GADTs}
  \begin{block}{}
    \begin{lstlisting}
data Term a where
  I :: Integer -> Term Integer
  B :: Bool -> Term Bool
  Add :: Term Integer -> Term Integer -> Term Integer
  Eq :: Term (?) -> Term (?) -> Term (?)
    \end{lstlisting}
  \end{block}
\end{frame}

\begin{frame}[fragile]
  \frametitle{Language definition with GADTs}
  \begin{block}{}
    \begin{lstlisting}
data Term a where
  I :: Integer -> Term Integer
  B :: Bool -> Term Bool
  Add :: Term Integer -> Term Integer -> Term Integer
  Eq :: (Eq a) => Term a -> Term a -> Term Bool
    \end{lstlisting}
  \end{block}
\end{frame}

\begin{frame}[fragile]
  \frametitle{Evaluation for GADTs}
  \begin{block}{}
    \begin{lstlisting}
eval :: Term a -> a -- This type annotation is mandatory
eval (I i) = i --$\pause$
eval (B b) = b --$\pause$
eval (Add t t') = eval t + eval t' --$\pause$
eval (Eq t t') = eval t == eval t' --$\pause$
    \end{lstlisting}
  \end{block}\pause

  We call this kind of interpreter ``tag-less'', because it does not
  require type tags like the data constructors \lstinline{Int} and \lstinline{Bool}.
\end{frame}

\begin{frame}[fragile]
  \frametitle{What about functions?}

  We want to add functions to our language.\pause

  \begin{block}{First try}
    \begin{lstlisting}
data FExp a where
  Var :: FExp a
  Lam :: FExp b -> FExp (a -> b)
  App :: FExp (a -> b) -> FExp a -> FExp b      
    \end{lstlisting}
  \end{block}\pause

  This doesn't work: not enough type information for variables and lambdas.
\end{frame}

\begin{frame}[fragile]
  \frametitle{Existential Types}

  In the definition of \lstinline{App}, \lstinline{a} is present in
  the argument types, but not in the result. Such a type variable
  stands for an \emph{existential type}: \\
  For each use of \lstinline{App} there is some type \lstinline{a} so
  that the types of the subterms work out.
  \begin{block}{}
    \begin{lstlisting}
  App :: FExp (a -> b) -> FExp a -> FExp b      
    \end{lstlisting}
  \end{block}\pause
  Demo!
\end{frame}

\begin{frame}[fragile]
  \frametitle{Back to functions!}
  \begin{block}{Type definition}
  \begin{lstlisting}
data FExp e a where
  App :: FExp e (a -> b) -> FExp e a -> FExp e b --$\pause$
  Lam :: FExp (a, e) b -> FExp e (a -> b) --$\pause$
  Var :: Nat e a -> FExp e a

data Nat e a where
  Zero :: Nat (a, b) a
  Succ :: Nat e a -> Nat (b, e) a
\end{lstlisting}
\end{block}\pause
Demo!
\end{frame}

\begin{frame}{Origin of GADTs}
  \begin{itemize}
  \item<1-> Comparatively recent extension to Haskell's type system.\\
    Invented by 3 different groups:
    \begin{itemize}
    \item Augustsson \& Petersson (1994): Silly Type Families
    \item Cheney \& Hinze (2003): First-Class Phantom Types.
    \item Xi, Chen \& Chen (2003): Guarded Recursive Datatype Constructors.
    \end{itemize}
  \item<2-> Type \emph{checking} is decidable.
  \item<3-> Type \emph{inference} is undecidable.
  \item<4-> Pattern matching is more complicated.
  \end{itemize}
  
\end{frame}

\begin{frame}[fragile]
  \frametitle{Wrapping up}
  \begin{itemize}[<+->]
  \item GADTs can express extra properties in types:
    \begin{itemize}
    \item 
    \end{itemize}
  \item We leverage Haskell's type system.
  \item GADTs do not solve \emph{all} the problems.
    For example, you can try to write a function of type
    \begin{lstlisting}
      parse :: String -> Expr a
    \end{lstlisting}
    \pause
    GADTs can be combined with other Haskell features such as type
    classes and type families. 
  \item GADTs become very complex when the domain grows!
  \end{itemize}
\end{frame}

\end{document}

%%% Local Variables:
%%% mode: latex
%%% TeX-master: t
%%% End:
