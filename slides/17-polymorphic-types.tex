%% -*- coding: utf-8 -*-
\documentclass[pdftex,aspectratio=169]{beamer}

%% -*- coding: utf-8 -*-
\usetheme{Boadilla} % default
\useoutertheme{infolines}
\setbeamertemplate{navigation symbols}{} 

\usepackage{etex}
\usepackage{alltt}
\usepackage{pifont}
\usepackage{color}
\usepackage[utf8]{inputenc}
%\usepackage{german}
\usepackage{listings}
\lstset{language=Haskell}
\lstset{sensitive=true}
\usepackage{hyperref}
\hypersetup{colorlinks=true}
\usepackage[final]{pdfpages}
\usepackage{url}
\usepackage{arydshln} % dashed lines
\usepackage{tikz}
\usepackage{mathpartir}

\DeclareUnicodeCharacter{3BB}{\ensuremath{\lambda}}


\newcommand\cmark{\ding{51}}
\newcommand\xmark{\ding{55}}

\newcommand{\nat}{\mathbf{N}}

\usepackage[all]{xy}

%% new arrow tip for xy
\newdir{|>}{!/4.5pt/@{|}*:(1,-.2)@^{>}*:(1,+.2)@_{>}}

\newcommand\cid[1]{\textup{\textbf{#1}}} % class names
\newcommand\kw[1]{\textup{\textbf{#1}}}  % key words
\newcommand\tid[1]{\textup{\textsf{#1}}} % type names
\newcommand\vid[1]{\textup{\texttt{#1}}} % value names
\newcommand\Mid[1]{\textup{\texttt{#1}}} % method names

\newcommand\TODO[1][]{{\color{red}{\textbf{TODO: #1}}}}

\newcommand\String[1]{\texttt{\dq{}#1\dq{}}}

\newcommand\ClassHead[1]{%
  \ensuremath{\begin{array}{|l|}
      \hline
      \cid{#1}
      \\\hline
    \end{array}}}
\newcommand\AbstractClass[2]{%
  \ensuremath{\begin{array}{|l|}
      \hline
      \cid{\textit{#1}}
      \\\hline
      #2
      \hline
    \end{array}}}
\newcommand\Class[2]{%
  \ensuremath{\begin{array}{|l|}
      \hline
      \cid{#1}
      \\\hline
      #2
      \hline
    \end{array}}}
\newcommand\Attribute[3][black]{\textcolor{#1}{\Param{#2}{#3}}\\}
\newcommand\Methods{\hline}
\newcommand\MethodSig[3]{\Mid{#2} (#3): \,\tid{#1}\\}
\newcommand\CtorSig[2]{\Mid{#1} (#2)\\}
\newcommand\AbstractMethodSig[3]{\Mid{\textit{#2}} (#3): \,\tid{#1}\\}
\newcommand\Param[2]{\vid{#2}:~\tid{#1}}

\lstset{%
  frame=single,
  xleftmargin=2pt,
  stepnumber=1,
  numbers=left,
  numbersep=5pt,
  numberstyle=\ttfamily\tiny\color[gray]{0.3},
  belowcaptionskip=\bigskipamount,
  captionpos=b,
  escapeinside={*'}{'*},
  % language=java,
  tabsize=2,
  emphstyle={\bf},
  commentstyle=\mdseries\it,
  stringstyle=\mdseries\rmfamily,
  showspaces=false,
  showtabs=false,
  keywordstyle=\bfseries,
  columns=fullflexible,
  basicstyle=\footnotesize\CodeFont,
  showstringspaces=false,
  morecomment=[l]\%,
  rangeprefix=////,
  includerangemarker=false,
}

\newcommand\CodeFont{\sffamily}

\definecolor{lightred}{rgb}{0.8,0,0}
\definecolor{darkgreen}{rgb}{0,0.5,0}
\definecolor{darkblue}{rgb}{0,0,0.5}

\newcommand\highlight[1]{\textcolor{blue}{\emph{#1}}}
\newcommand\GenClass[2]{\cid{#1}\texttt{<}\cid{#2}\texttt{>}}

\newcommand\Colored[3]{\alt<#1>{\textcolor{#2}{#3}}{#3}}

\newcommand\nt[1]{\ensuremath{\langle#1\rangle}}

\newcommand{\free}{\operatorname{free}}
\newcommand{\bound}{\operatorname{bound}}
\newcommand{\var}{\operatorname{var}}
\newcommand\VSPBLS{\vspace{-\baselineskip}}

\newcommand\IF{\textit{IF}}
\newcommand\TRUE{\textit{TRUE}}
\newcommand\FALSE{\textit{FALSE}}

\newcommand\IFZ{\textit{IF0}}
\newcommand\ZERO{\textit{ZERO}}
\newcommand\SUCC{\textit{SUCC}}
\newcommand\ADD{\textit{ADD}}
\newcommand\SUB{\textit{SUB}}
\newcommand\MULT{\textit{MULT}}
\newcommand\DIV{\textit{DIV}}

\newcommand\PAIR{\textit{PAIR}}
\newcommand\FST{\textit{FST}}
\newcommand\SND{\textit{SND}}

\newcommand\CASE{\textit{CASE}}
\newcommand\LEFT{\textit{LEFT}}
\newcommand\RIGHT{\textit{RIGHT}}

\newcommand\Encode[1]{\lceil#1\rceil}
\newcommand\Reduce{\stackrel\ast\rightarrow_\beta}

\newcommand\Nat{\textit{Nat}}
\newcommand\Bool{\textit{Bool}}
\newcommand\Pair{\textit{Pair}}
\newcommand\Tfun[1]{#1\to}

\newcommand\Tenv{A}
\newcommand\Lam[1]{\lambda#1.}
\newcommand\App[1]{#1\,}
\newcommand\Succ{\textit{SUCC}\,}
\newcommand\Let[2]{\textit{let}\,#1=#2\,\textit{in}\,}

\newcommand\calE{\mathcal{E}}
\newcommand\calU{\mathcal{U}}
\newcommand\calP{\mathcal{P}}
\newcommand\calW{\mathcal{W}}

\newcommand\GEN{\textit{gen}}
\newcommand\EFV[1]{\textit{fv} (#1)}
\newcommand\Dom[1]{\textit{dom} (#1)}

%%% Local Variables: 
%%% mode: latex
%%% TeX-master: nil
%%% End: 

%%% frontmatter
%% -*- coding: utf-8 -*-

\title{Functional Programming}
\subtitle{Introduction}

\author[Peter Thiemann]{Prof. Dr. Peter Thiemann}
\institute[Univ. Freiburg]{Albert-Ludwigs-Universität Freiburg, Germany}
\date{SS 2021}


\subtitle
{Polymorphic Types}

\begin{document}

\begin{frame}
  \titlepage
\end{frame}

%\begin{frame}
%  \frametitle{Outline}
%  \tableofcontents
  % You might wish to add the option [pausesections]
%\end{frame}


\section{ML-Style Polymorphic Types}
\begin{frame}
  \frametitle{ML-Style Polymorphic Types}
  \begin{block}<+->{Simple Types}
    restrictive, insufficient modularity
  \end{block}
  \begin{exampleblock}<.->{Example}
    \begin{displaymath}
      \App{(\Lam{i} \App{(\App{i} (\Lam{y}\Succ y))} (\App{i}42))} (\Lam{x}x)
    \end{displaymath}
    \VSPBLS
    \begin{itemize}
    \item Simple typing derives $\cdot \vdash \Lam{x}x : \alpha\to\alpha$
    \item $\App{i}42$ requires $i : \Nat \to \beta$
    \item $\App{i} (\Lam{y}\Succ y)$ requires $i : (\Nat\to\Nat) \to \gamma$
    \item Unification of the assumptions on $i$ fails: term has no
      simple type
    \item However, term evaluates without error
    \end{itemize}
  \end{exampleblock}
  \begin{alertblock}<+->{Approach: Parametric polymorphism $\Lam xx : \textcolor{black}{\forall \alpha}.\alpha\to\alpha$}
    
  \end{alertblock}
\end{frame}

\begin{frame}
  \frametitle{Applied Mini-ML}
  \vspace{-\baselineskip}
  \begin{block}{Syntax}
    %\vspace{-1.5\baselineskip}
    \begin{displaymath}
      \begin{array}{rlcl}
        \mathsf{Exp} \ni & e,f & ::= & x \mid \Lam{x}e \mid \App{f}e
        \mid \Let{x}{e}f \mid n \mid \Succ e \\
        \mathsf{Val} \ni & v & ::= & \Lam{x}e \mid n
      \end{array}
    \end{displaymath}
  \vspace{-1\baselineskip}
  \end{block}
  \begin{block}{Evaluation (Call-by-Value)}
    % \vspace{-1.5\baselineskip}\small
    \begin{mathpar}
      \inferrule[Beta-V]{}{\App{(\Lam{x}e)}{v} \rightarrow_v e[x \mapsto v]}

      \inferrule[AppL]{f \rightarrow_v f'}{ \App{f}{e} \rightarrow_v \App{f'}{e}}

      \inferrule[VAppR]{e \rightarrow_v e'}{ \App{v}{e} \rightarrow_v \App{v}{e'}}

      \inferrule[LetL]{e \rightarrow_v e'}{\Let{x}{e}f \rightarrow_v
        \Let{x}{e'}f}

      \inferrule[Beta-Let]{}{\Let{x}{v}e \rightarrow_v e[x \mapsto v]}

      \inferrule[SuccL]{e \rightarrow_v e'}{ \Succ{e} \rightarrow_v \Succ{e'}}

      \inferrule[Delta]{e \rightarrow_\delta e'}{ e \rightarrow_v {e'}}
    \end{mathpar}
    \vspace{-1\baselineskip}
  \end{block}
\end{frame}

\begin{frame}
  \frametitle{Types for Applied Mini-ML}
  \begin{block}{Syntax of Types}
    \begin{displaymath}
      \begin{array}{lcll}
        \tau & ::= & \alpha \mid \tau \to \tau \mid \Nat &
        \textrm{Types}\\\color{red}
        \sigma & ::= & \color{red} \tau \mid \forall \alpha. \sigma & 
        \color{red}\textrm{Type Schemes}\\
        \Tenv & ::= & \cdot \mid \Tenv, x:\color{red}\sigma &
        \textrm{Type Environments}
      \end{array}
    \end{displaymath}
  \end{block}
  A \textbf{type scheme  $\forall\alpha.\sigma$} \dots
  \begin{itemize}
  \item \emph{binds} type variable $\alpha$
  \item can be \emph{instantiated} by substituting a type for $\alpha$
    in $\sigma$
  \item only appears in the type environment
  \item restricts introduction of type variables to toplevel!
  \end{itemize}
\end{frame}

\begin{frame}
  \frametitle{Operations on Type Schemes}
  \begin{block}<+->{Generic Instance}
    $\sigma = \forall\alpha_1\dots\alpha_m.\tau$ has a \textbf{generic
      instance}
    $\sigma' = \forall\beta_1\dots\beta_n.\tau'$, written as 
    $\sigma \succeq \sigma'$, if for all $i$, $\beta_i\notin
    \EFV{\sigma}$ and there is a substitution $S$ with
    $\Dom{S} \subseteq \{\alpha_1,\dots,\alpha_m\}$ such that $\tau' = S\tau$.
  \end{block}
  \begin{exampleblock}<+->{Examples}\VSPBLS
    \begin{align*}
      \forall\alpha.\alpha\to\alpha &\succeq \Nat\to\Nat
      &
        \forall\alpha\beta.\alpha\to\beta\to\alpha
      & \succeq \forall \alpha. \alpha\to\alpha\to \alpha
      \\
      \forall\alpha.\alpha\to\beta\to\alpha & \succeq \beta\to\beta\to\beta
      &
      \forall\alpha.\alpha\to\beta\to\alpha & \succeq \Nat\to\beta\to\Nat'
    \end{align*}
  \end{exampleblock}
  \begin{block}<+->{Generalization}
    \begin{displaymath}
      \GEN (\Tenv , \tau ) = \forall \alpha_1 \dots \alpha_m. \tau
    \end{displaymath}
    where $\{ \alpha_1,\dots, \alpha_m \} = \EFV{\tau} \setminus \EFV{\Tenv}$.
    \qquad
    $\alpha_1,\dots,\alpha_m$ are \textbf{generic variables} in $\tau$.
  \end{block}
\end{frame}

\begin{frame}
  \frametitle{Inference Rules for Mini-ML}
  \framesubtitle{syntax-directed}
  \vspace{-\baselineskip}
  \begin{mathpar}
    \inferrule[Var]{\color{red}\sigma \succeq \tau}{\Tenv, x:\sigma \vdash x : \tau}

    \inferrule[Lam]{
      \Tenv, x : \tau \vdash e : \tau'
    }{
      \Tenv \vdash \Lam{x} e : \Tfun{\tau}{\tau'}}
    
    \inferrule[App]{
      \Tenv \vdash e : \Tfun{\tau}{\tau'} \\
      \Tenv \vdash f : \tau
    }{
      \Tenv \vdash \App{e}{f} : \tau'
    }

    \color{red}
    \inferrule[Let]{
      \Tenv \vdash e : \tau \\
      \Tenv, x : \GEN (\Tenv, \tau) \vdash f : \tau'
    }{
      \Tenv \vdash \Let{x}{e}{f} : \tau'
    }
    \color{black}
    \\
    \inferrule[Num]{}{\Tenv \vdash n : \Nat}

    \inferrule[Succ]{\Tenv \vdash e : \Nat}{\Tenv \vdash \Succ e :
      \Nat
    }
  \end{mathpar}
\end{frame}


\begin{frame}
  \frametitle{Example Revisited}
  \begin{displaymath}
    \Let{i}{\Lam{x}x} \App{(\App{i} (\Lam{y}\Succ y))} (\App{i}42)
  \end{displaymath}
  \begin{itemize}
  \item $\cdot\vdash\Lam{x}x : \alpha\to\alpha$
  \item $\GEN (\cdot, \alpha\to\alpha) =\forall \alpha. \alpha \to\alpha$
  \item Generalized binding:  $i : \forall \alpha. \alpha \to\alpha$
  \item $\App{i}42$ using instance $\forall \alpha. \alpha \to\alpha \succeq \Nat\to\Nat$
  \item $\App{i} (\Lam{y}\Succ y)$ using instance $\forall \alpha. \alpha \to\alpha \succeq (\Nat \to\Nat) \to
    (\Nat \to\Nat)$
  \item Type checking succeeds
  \item Type checking the uses of $i$ is better decoupled from $i$'s
    definition $\Rightarrow$ modularity improved
  \end{itemize}
\end{frame}

\begin{frame}
  \frametitle{Properties}
  \begin{itemize}
  \item Type soundness
  \item Decidable type checking and type inference (upcoming)
  \item Basis for type system of ML, Haskell, and other languages
  \item Numerous extensions
  \end{itemize}
\end{frame}

\section{Type Inference for Mini-ML}
\begin{frame}
    \Huge
  \begin{center}
    {Type Inference for Mini-ML}
  \end{center}
\end{frame}

\begin{frame}
  \frametitle{Type Inference for Mini-ML}
  \begin{block}<+->{Hindley-Milner Type Inference Algorithm $\calW (\Tenv; e)$}
    transforms a type environment
    $\Tenv$ and a  term $e$ into a pair $(S, \tau)$ of a
    substitution and a type (or fails if no typing exists).

    See: Milner, Robin (1978). A Theory of Type Polymorphism in Programming. JCSS, 17: 348–375
  \end{block}
  \begin{block}<+->{Notation}
    \begin{itemize}
    \item $\mathbf{fresh}$ creates one or more fresh type variables, which are not yet in use
    \item $\mathit{ID}$ the identity substitution
    \item $S$ and $U$ range over type substitutions
    \end{itemize}
  \end{block}
\end{frame}

\begin{frame}
  \frametitle{Mini-ML Type Inference Algorithm, Part I}
  %\vspace{-2\baselineskip}
  \begin{displaymath}
    \begin{array}{lcl}
      \calW (\Tenv ; x) &=& \mathbf{let}\
      \forall\alpha_1\dots\alpha_m.\tau = \Tenv (x) \\
      && \beta_1\dots\beta_m \gets \mathbf{fresh} \\
      && \mathbf{return}\ (\mathit{ID}, \tau[\alpha_i\mapsto\beta_i])
      \\
      \calW (\Tenv ; \Lam{x}e) &=&
      \beta \gets \mathbf{fresh} \\
      &&  (S, \tau) \gets \calW (\Tenv, x: \beta; e) \\
      && \mathbf{return}\ (S, S\beta \to \tau)
      \\
      \calW (\Tenv; \App{e_0}{e_1}) &=&
      (S_0, \tau_0) \gets \calW (\Tenv ; e_0) \\
      && (S_1, \tau_1) \gets \calW (S_0\Tenv ; e_1)\\
      &&\beta \gets \mathbf{fresh}\\
      &&U \gets \calU (S_1\tau_0 \doteq \tau_1\to \beta)\\
      &&\mathbf{return}\ (U \circ S_1 \circ S_0, U\beta)
      \\
      \calW (\Tenv; \Let{x}{e_0}{e_1}) &=&
      (S_0, \tau_0) \gets \calW (\Tenv ; e_0)\\
      && \mathbf{let}\ \sigma = \GEN (S_0\Tenv, \tau_0) \\
      && (S_1, \tau_1) \gets \calW (S_0\Tenv, x :\sigma; e_1) \\
      &&\mathbf{return}\ (S_1\circ S_0, \tau_1) \\
    \end{array}
  \end{displaymath}
\end{frame}

\begin{frame}
  \frametitle{Mini-ML Type Inference Algorithm, Part II}
 \begin{displaymath}
   \begin{array}{lcl}
      \calW (\Tenv ;  n) &=&
      \mathbf{return}\ (\mathit{ID}, \Nat) \\
      \calW (\Tenv ; \Succ e) &=&
      (S, \tau) \gets \calW (\Tenv ; e) \\
      &&\mathbf{let}\ U \gets \calU (\tau\doteq\Nat)\ \mathbf{in} \\
      &&\mathbf{return}\ (U \circ S, \Nat)
   \end{array}
 \end{displaymath}
\end{frame}


\begin{frame}
  \frametitle{Properties of Type Inference for Mini-ML}
  \begin{block}{Soundness}
    If $\calW (\Tenv; e) = \mathbf{return}\ (S, \tau)$, then $S\Tenv \vdash e : \tau$.
  \end{block}
  \begin{block}{Completeness}
    If $S\Tenv \vdash e : \tau'$, then $\calW (\Tenv ; e) =
    \mathbf{return}\ (T, \tau)$ such that $S = S'\circ T$ and $\tau'= S'\tau$.
  \end{block}
  \begin{block}{Principal types}
    Completeness implies that $\calW$ computes \textbf{principal types} because all other types of the same term are instances of the computed type.
  \end{block}
\end{frame}


\begin{frame}
  \frametitle{Wrapup}
  \begin{itemize}
  \item ML polymorphism is based on type schemes
  \item Type checking and inference is decidable
  \item Type inference yields a principal type
  \end{itemize}
\end{frame}
\end{document}


